\PassOptionsToPackage{unicode}{hyperref}
\PassOptionsToPackage{naturalnames}{hyperref}
\documentclass[12pt, xetex, xcolor=pdftex, dvipsnames]{beamer}
\usetheme{metropolismyfont}
\usepackage{dcolumn} % for apsrtable outputs
\usepackage{xunicode} % extra support for unicode
\usepackage{xltxtra} % implements some odds-and-ends features
\usepackage{verbatim} % for multiple-line comments
% to fix a problem in line-breaks
% see "http://zrbabbler.hp.infoseek.co.jp/xelatex.html"
\XeTeXlinebreaklocale "ja"
\XeTeXlinebreakskip=0pt plus 1pt
\XeTeXlinebreakpenalty=0
\def\<{\@ifstar{\zx@hwback\nobreak}{\zx@hwback\relax}}
\def\zx@hwback#1{\leavevmode#1\hskip-.5em\relax}

\usepackage{color}
\usepackage{listings}
\lstset{%
    language={Python},%
    basicstyle={\footnotesize},%
    identifierstyle={\footnotesize},%
    commentstyle={\footnotesize\color{mLightGreen}},%
    keywordstyle={\footnotesize\color{mLightBrown}},%
    stringstyle={\footnotesize\color{mDarkBrown}},%
    frame={tblr}%
}

\AtBeginSection[]{
    \begin{frame}{目次}
        \setbeamertemplate{section in toc}[sections numbered]
        \tableofcontents[currentsection, hideallsubsections]
    \end{frame}
}

\title{引き継ぎ資料 Vol.3}
\subtitle{CにないPythonの世界}
\date{2016/07/??}
\author{}
\institute{}
\begin{document}
\maketitle
\begin{frame}{コンセプト}
    Cには無いPythonの世界を堪能しよう!
\end{frame}
\begin{frame}{目次}
  \setbeamertemplate{section in toc}[sections numbered]
  \tableofcontents[hideallsubsections]
\end{frame}

\section{タプル・リスト・ディクショナリ}
\begin{frame}{Cの言語仕様にはないデータ構造}
    \begin{itemize}
        \item タプル
        \item リスト
        \item ディクショナリ
    \end{itemize}
\end{frame}
\subsection{タプル}
\begin{frame}{タプルとは}
    数が並んだもの

    Cで近い機能は配列
\end{frame}
\begin{frame}[fragile]{タプル}
    簡単な例
    \begin{lstlisting}
>>> t = (1, 2, 3)
>>> print(t)
(1, 2, 3)
>>> print(t[0])
1
    \end{lstlisting}
    \only<+->{ここまではCの配列と同じ}
\end{frame}
\begin{frame}[fragile]{タプル}
    あえてCで書くなら...
    \begin{lstlisting}[language={C}]
#include <stdio.h>
#include <stdlib.h>

int main(int argc, char** argv) {
    int t[] = { 1, 2, 3 };
    printf("%d\n", t[0]);
    return 0;
}
    \end{lstlisting}
    簡単!
\end{frame}
\begin{frame}[fragile]{難しい例}
\begin{lstlisting}
>>> t = 1, 2, 3
>>> x, y, z = t
>>> def hoge():
...     return 4, 5, 6
...
>>> a1, a2, a3 = hoge()
>>> a = hoge()
>>> u, v, w = z, y, x
\end{lstlisting}
\only<+->{それぞれの変数の中身は?}
\end{frame}
\begin{frame}[fragile]{答}
    \begin{columns}[t]
        \begin{column}{0.49\hsize}
\begin{lstlisting}
>>> t = 1, 2, 3
>>> x, y, z = t
>>> def hoge():
...     return 4, 5, 6
...
>>> a1, a2, a3 = hoge()
>>> a = hoge()
>>> u, v, w = z, y, x
\end{lstlisting}
        \end{column}
        \begin{column}{0.49\hsize}
\begin{lstlisting}
>>> print(t)
(1, 2, 3)
>>> print(x, y, z)
1 2 3
>>> print(a1, a2, a3)
4 5 6
>>> print(a)
(4, 5, 6)
>>> print(u, v, w)
3 2 1
\end{lstlisting}
        \end{column}
    \end{columns}
\end{frame}
\begin{frame}{タプルの要点}
    \begin{itemize}
        \item タプルに必要なのは``,''(カンマ)
        \item 複数の値を返す関数はタプルを一つ返す関数
        \item タプルは自動的に展開され複数の変数に代入
        \item \alert{タプルでは要素の変更は不可}
    \end{itemize}
\end{frame}
\begin{frame}[fragile]
    \alert{タプルでは要素の変更は認められない}
\begin{lstlisting}
>>> t = (1, 2, 3)
>>> t[1] = 4
Traceback (most recent call last):
  File "<stdin>", line 1, in <module>
TypeError: 'tuple' object does not support item
assignment
\end{lstlisting}
\end{frame}

\subsection{リスト}
\begin{frame}{リストとは}
    オブジェクト(整数,小数,その他)が並んだもの

    後から変更が可能
\end{frame}
\begin{frame}[fragile]{例}
    \begin{lstlisting}
>>> t = []
>>> t.append(1)
>>> t.append(2)
>>> print(t)
[1, 2]
>>> t[1] = 3
>>> print(t)
[1, 3]
>>> lst = [1, 3, 5, 7, 9]
>>> print(lst)
[1, 3, 5, 7, 9]
>>> [a, b] = [1, 3]
    \end{lstlisting}
\end{frame}
\begin{frame}[fragile]{便利な例}
    \begin{lstlisting}
>>> t = [1, 3, 5, 7, 9]
>>> print(t[1:4])
[3, 5, 7]
>>> print(len(t))
5
>>> s = [2, 4, 6, 8]
>>> print(t + s)
[1, 3, 5, 7, 9, 2, 4, 6, 8]
>>> print(t * 2)
[1, 3, 5, 7, 9, 1, 3, 5, 7, 9]
>>> print(2 in t, 3 in t)
False True
    \end{lstlisting}
\end{frame}
\begin{frame}[fragile]{リスト}
    難しい例
\begin{lstlisting}
>>> t = [1, 2, 3, 4, 5]
>>> def hoge(x):
...     return x**2
... 
>>> xs = [hoge(x) for x in t]
\end{lstlisting}
\end{frame}

\subsection{ディクショナリ}
\begin{frame}{ディクショナリとは}
    添字に数字以外が使えるリスト

    言語によっては連想配列・Map・HashMapなど  
\end{frame}
\begin{frame}[fragile]{簡単な例}
    \begin{lstlisting}
>>> d = {"a": 1, "b": 2, "c": 3}
>>> print(d["a"])
1
>>> d["b"] = 5
>>> print(d)
{'b': 5, 'c': 3, 'a': 1}
    \end{lstlisting}
\end{frame}
\begin{frame}{注意点}
    \begin{itemize}
        \item 存在しない要素を取得しようとするとエラー
        \item 存在しない要素に代入すると要素を追加
    \end{itemize}
\end{frame}
\subsection{まとめ}
\begin{frame}{まとめ}
    タプル・リスト・ディクショナリはCには無い

    Pythonにはデフォルトで存在

    お互いにネスト(入れ子)が可能

    \alert{Cよりも柔軟に複雑なデータ構造を表現可能}
\end{frame}

\section{関数}
\subsection{Cの関数の概念}
\begin{frame}
    \alert{\Large Cの関数どこまでマスターしてますか?}
\end{frame}
\begin{frame}[fragile]{Cの関数}
    \begin{lstlisting}[language={C}]
int add(int a, int b) {
    return a + b;
}
void swap(int* a, int* b) {
    int tmp = *a;
    *a = *b;
    *b = tmp;
}
void map(int* array, int N, int (*func)(int)) {
    for (int i = 0; i < N; ++i)
        array[i] = func(array[i]);
}
    \end{lstlisting}
\end{frame}
\begin{frame}{Cで出てきた概念}
    \begin{itemize}
        \item 引数
        \item 戻り値
        \item 値渡し・参照渡し
        \item 関数ポインタ
    \end{itemize}

    \pause
    Pythonはもう少し難しい概念を持つ
\end{frame}
\subsection{Python関数の基本}
\begin{frame}[fragile]{Python関数の基本}
\begin{lstlisting}
>>> def add(a, b):
...     return a + b
... 
>>> print(add(1, 3))
4
\end{lstlisting}
\pause
\begin{description}
\item[add] 関数名
\item[a, b] 引数
\item[return a + b] a + bを戻り値とする
\item[add(1, 3)] 関数呼び出し
\end{description}
\end{frame}
\begin{frame}{Pythonの関数が持つ概念}
    \begin{itemize}
        \item 引数
        \begin{itemize}
            \item 参照の値渡し
            \item キーワード引数
        \end{itemize}
        \item 戻り値
        \item 関数オブジェクト
        \begin{itemize}
            \item lambda式
        \end{itemize}
    \end{itemize}
\end{frame}
\subsection{引数}
\begin{frame}{引数}
    関数処理の材料

    Cの場合二種類:
    \begin{description}
        \item[値渡し] 変数をそのまま渡す
        \item[参照渡し] 変数のポインタを渡す
    \end{description}

    Pythonは全て\alert{参照の値渡し}

    \pause
    これがとってもややこしい
\end{frame}
\begin{frame}[fragile]{参照の値渡し}
    \begin{columns}[t]
        \begin{column}{0.51\hsize}
\begin{lstlisting}
>>> def swap(a, b):
...     tmp = a
...     a = b
...     b = tmp
...
>>> def swap_e(lst, i, j):
...     tmp = lst[i]
...     lst[i] = lst[j]
...     lst[j] = tmp
...
\end{lstlisting}
        \end{column}
        \begin{column}{0.47\hsize}
\begin{lstlisting}
>>> a, b = 1, 2
>>> swap(a, b)
>>> print(a, b)
1 2
>>> lst = [1, 3, 5, 7]
>>> swap_e(lst, 2, 3)
>>> print(lst)
[1, 3, 7, 5]
\end{lstlisting}
        \end{column}
    \end{columns}
\end{frame}
\begin{frame}{参照の値渡し}
    Pythonの引数はローカル変数に\alert{代入}される

    \begin{block}{不変性のオブジェクト(整数・小数・タプル\dots)}
        \begin{itemize}
            \item Cの値渡しと似た動作
            \item 関数内での書き換えは呼び出し元に影響しない
        \end{itemize}
     \end{block}
    \begin{block}{可変性のオブジェクト(リスト・ディクショナリ\dots)}
        \begin{itemize}
            \item Cの参照渡しと似た動作
            \item 関数内での書き換えが呼び出し元に影響する
        \end{itemize}
    \end{block}

    \pause
    \alert{思わぬバグの可能性あり}
\end{frame}
\begin{frame}[fragile]{高度な引数の指定}
ライブラリのリファレンスでまれに見る形式

\begin{lstlisting}
def func(*args, **kwargs)
\end{lstlisting}

こいつの*args,**kwargsって何?って話
\end{frame}
\begin{frame}{引数の形式}
    どちらも\alert{可変長引数}を実現

    \begin{block}{*args}
        \begin{itemize}
            \item 位置で指定された引数が格納
            \item 引数がタプル形式で格納
        \end{itemize}
    \end{block}
    \begin{block}{**kwargs}
        \begin{itemize}
            \item キーワードと共に指定された引数が格納
            \item 引数がディクショナリ形式で格納
        \end{itemize}
    \end{block}
\end{frame}
\begin{frame}[fragile]{例}
\begin{lstlisting}
>>> def hoge(*args, **kwargs):
...     print(args)
...     print(kwargs)
...
>>> hoge(1, "a", cat="cute", dog="mustdie")
(1, 'a')
{'dog': 'mustdie', 'cat': 'cute'}
\end{lstlisting}
\pause

リファレンスに明確に書かれていない引数も渡せる
\end{frame}
\subsection{戻り値}
\begin{frame}{戻り値}
    基本的にはCと同じ

    ただし...
    \begin{itemize}
        \item 複数の値を返す事が可能
        \item return文を書かなくても関数はNoneを返す
    \end{itemize}
\end{frame}
\subsection{関数オブジェクト}
\begin{frame}{関数オブジェクト}
    Pythonでは関数もオブジェクトの一つ

    いろんなことが可能
    \begin{itemize}
        \item 変数に代入
        \item 引数として渡す
    \end{itemize}

    Cの関数ポインタに近い
\end{frame}
\begin{frame}[fragile]{例-変数に格納-}
\begin{lstlisting}
>>> def square(x):
...     return x ** 2
... 
>>> square_func = square
>>> print(square_func(2))
4
\end{lstlisting}
\end{frame}
\begin{frame}[fragile]{例-引数-}
\begin{lstlisting}
>>> def mapping(seq, func):
...     mapped = []
...     for e in seq:
...         mapped.append(func(e))
...     return mapped
... 
>>> def square(x):
...     return x ** 2
... 
>>> t = [1, 3, 5]
>>> t_squared = mapping(t, square)
>>> print(t_squared)
[1, 9, 25]
\end{lstlisting}
\end{frame}
\begin{frame}{何に使うのか?}
    例えば...
    \begin{itemize}
        \item 条件に合うものを抽出したい
        \begin{itemize}
            \item 条件とリストをそれぞれ格納
            \item 両方をfilter関数の引数に
        \end{itemize}
        \item 積分をしたい
        \begin{itemize}
            \item 被積分関数を作る
            \item scipy.integrate.quad関数に渡す
        \end{itemize}
        \item 極値を探したい
        \begin{itemize}
            \item 微分した関数を作る
            \item scipy.optimize.root関数に渡す
        \end{itemize}
    \end{itemize}

    数学的なことを考える時に相性が良い(気がする)
\end{frame}
\begin{frame}{lambda式}
    いちいち関数を外部に定義するの面倒

    そんなあなたにlambda式
\end{frame}
\begin{frame}[fragile]{例}
ちょっとした関数なら簡単にかける

\begin{lstlisting}
>>> square_func = lambda x: x ** 2
\end{lstlisting}

\only<2>{\alert{複雑な関数は書くと読みづらいので注意}}
\end{frame}
\section{オブジェクト}

\section{名前空間}

\section{numpy・matplotlib}

\end{document}