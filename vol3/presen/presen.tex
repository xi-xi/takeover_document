\PassOptionsToPackage{unicode}{hyperref}
\PassOptionsToPackage{naturalnames}{hyperref}
\documentclass[12pt, xetex, xcolor=pdftex, dvipsnames]{beamer}
\usetheme{metropolismyfont}
\usepackage{dcolumn} % for apsrtable outputs
\usepackage{xunicode} % extra support for unicode
\usepackage{xltxtra} % implements some odds-and-ends features
\usepackage{verbatim} % for multiple-line comments
% to fix a problem in line-breaks
% see "http://zrbabbler.hp.infoseek.co.jp/xelatex.html"
\XeTeXlinebreaklocale "ja"
\XeTeXlinebreakskip=0pt plus 1pt
\XeTeXlinebreakpenalty=0
\def\<{\@ifstar{\zx@hwback\nobreak}{\zx@hwback\relax}}
\def\zx@hwback#1{\leavevmode#1\hskip-.5em\relax}

\usepackage{color}
\usepackage{listings}
\lstset{%
    language={Python},%
    basicstyle={\footnotesize},%
    identifierstyle={\footnotesize},%
    commentstyle={\footnotesize\color{mLightGreen}},%
    keywordstyle={\footnotesize\color{mLightBrown}},%
    stringstyle={\footnotesize\color{mDarkBrown}},%
    frame={tblr}%
}

\title{引き継ぎ資料 Vol.3}
\subtitle{CにないPythonの世界}
\date{2016/07/??}
\author{}
\institute{}
\begin{document}
\maketitle
\begin{frame}{コンセプト}
    Cには無いPythonの世界を堪能しよう!
\end{frame}
\begin{frame}{目次}
  \setbeamertemplate{section in toc}[sections numbered]
  \tableofcontents[hideallsubsections]
\end{frame}

\section{タプル・リスト・ディクショナリ}
\begin{frame}{Cの言語仕様にはないデータ構造}
    \begin{itemize}
        \item タプル
        \item リスト
        \item ディクショナリ
    \end{itemize}
\end{frame}
\subsection{タプル}
\begin{frame}[fragile]{タプル}
    簡単な例
    \begin{lstlisting}
>>> t = (1, 2, 3)
>>> print(t)
(1, 2, 3)
>>> print(t[0])
1
    \end{lstlisting}
    \only<+->{ここまではCの配列と同じ}
\end{frame}
\begin{frame}[fragile]{タプル}
    あえてCで書くなら...
    \begin{lstlisting}[language={C}]
#include <stdio.h>
#include <stdlib.h>

int main(int argc, char** argv) {
    int t[] = { 1, 2, 3 };
    printf("%d\n", t[0]);
    return 0;
}
    \end{lstlisting}
    簡単!
\end{frame}
\begin{frame}[fragile]{難しい例}
\begin{lstlisting}
>>> t = 1, 2, 3
>>> x, y, z = t
>>> def hoge():
...     return 4, 5, 6
...
>>> a1, a2, a3 = hoge()
>>> a = hoge()
>>> u, v, w = z, y, x
\end{lstlisting}
\only<+->{それぞれの変数の中身は?}
\end{frame}
\begin{frame}[fragile]{答}
    \begin{columns}[t]
        \begin{column}{0.49\hsize}
\begin{lstlisting}
>>> t = 1, 2, 3
>>> x, y, z = t
>>> def hoge():
...     return 4, 5, 6
...
>>> a1, a2, a3 = hoge()
>>> a = hoge()
>>> u, v, w = z, y, x
\end{lstlisting}
        \end{column}
        \begin{column}{0.49\hsize}
\begin{lstlisting}
>>> print(t)
(1, 2, 3)
>>> print(x, y, z)
1 2 3
>>> print(a1, a2, a3)
4 5 6
>>> print(a)
(4, 5, 6)
>>> print(u, v, w)
3 2 1
\end{lstlisting}
        \end{column}
    \end{columns}
\end{frame}
\begin{frame}{タプルの要点}
    \begin{itemize}
        \item タプルに必要なのは``,''(カンマ)
        \item 複数の値を返す関数はタプルを一つ返す関数
        \item タプルは自動的に展開され複数の変数に代入
        \item \alert{タプルでは要素の変更は不可}
    \end{itemize}
\end{frame}
\begin{frame}[fragile]
    \alert{タプルでは要素の変更は認められない}
\begin{lstlisting}
>>> t = (1, 2, 3)
>>> t[1] = 4
Traceback (most recent call last):
  File "<stdin>", line 1, in <module>
TypeError: 'tuple' object does not support item
assignment
\end{lstlisting}
\end{frame}

\subsection{リスト}
\begin{frame}[fragile]{リスト}

\end{frame}

\section{関数}

\section{オブジェクト}

\section{名前空間}

\section{numpy・matplotlib}

\end{document}